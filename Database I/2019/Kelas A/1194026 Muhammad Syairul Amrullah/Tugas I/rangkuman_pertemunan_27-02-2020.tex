\documentclass[a4paper,12pt]{article}
    \usepackage[utf8]{inputenc}
    \usepackage{color}
    \usepackage{graphicx}
    \usepackage{ragged2e}
    \newcommand{\hilight}[1]{\colorbox{yellow}{#1}}

    \begin{document}
        \begin{center}
            \includegraphics[width=5cm,height=5cm]{D:/testingg/logo-poltekpos.png}
        \end{center}
        \begin{center}
            \textbf{Basis  Data I} \\
        \end{center}
        \vspace{-0.6cm}
        \begin{center}
            \begin{tabular}{ c c l}
                Nama & : & Muhammad Syiarul Amrullah\\
                Prodi & : & D4 Teknik Informatika\\
                Kelas & : & 1A
            \end{tabular}
        \end{center}
        \vspace{0.6cm}
        \begin{center}
            Untuk Memenuhi tugas Basis Data I\\
        \end{center}
        \begin{center}
          \textbf{Dosen Pengampu}\\
          \textbf{Syafrial Fachri Pane, S.T., M.T.I.,EBDP}
        \end{center}
        \vspace{0.5cm}
        \begin{center}
            \textbf{POLITEKNIK POS INDONESIA}\\
            \textbf{D4 TEKNIK INFORMATIKA}\\
            \textbf{2019/2020}
        \end{center}
        \newpage
        \begin{center}
          \huge{\textbf{Rangkuman Basis Data I\\Pertemuan 27-02-2020}} %textbf = bold
        \end{center}
         \begin{figure}
            \centering
            \includegraphics[width=5cm]{D:/tugas_dokumen/database/tugas 1/database&cloud.png}
          \end{figure}
        \section{\textbf{pengenalan basis data}}

         \par Basis data merupakan kumpulan beberapa informasi yang berupa angka, bilangan, huruf, DLL. dalam basis data, informasi yang ada di dalam nya harus valid dan berasal dari sumber data yang valid dan basis data juga berfungsi sebagai tempat penyimpanan data dengan terstruktur dan melindungi data yang ada di dalamnya dengan baik.
        \vspace{-0.5cm}
        \section{\textbf{Penjelasan mengenai DBMS dan RDBMS}}
        \begin{enumerate}
          \item \textbf{DBMS(Database Management System)}\\
          \vspace{-0.5cm}
          \par sebelumnya sudah di jelaskan bahwa database bisa dijadikan sebagai tempat penyimpanan data, dapat mengatur data sehingga tidak terjadi pengulangan data dan menjaga keamanan data. untuk melakukan hal itu Database mempunyai yang disebut \textbf{DBMS(Database Management System)} yang merupakan suatu software yang dibuat khusus untuk mengelola suatu database. macam - macam software database:\\
          \vspace{-0.5cm}
          \begin{enumerate}
            \item mysql
            \item mongo db
            \item oracle
          \end{enumerate}
          \newpage
          \item \textbf{RDBMS(Relational Database Management System)}\\
          \vspace{-0.7cm}
          \justify dalam suatu database juga memiliki banyak table yang berhubungan satu sama lain, sehingga seluruh data yang ada di dalam semua table saling berintegrasi, ini yang disebut dengan \textbf{RDBMS(Relational Database Management System)}
        \end{enumerate}
    \end{document}




